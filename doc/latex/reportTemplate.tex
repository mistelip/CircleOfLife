\documentclass[11pt]{article}
\usepackage{geometry}                
\geometry{letterpaper}                   

\usepackage{graphicx}
\usepackage{amssymb}
\usepackage{epstopdf}
\usepackage{natbib}
\usepackage{amssymb, amsmath}
\usepackage{enumitem}
\setlist[description]{leftmargin=\parindent,labelindent=\parindent}
\DeclareGraphicsRule{.tif}{png}{.png}{`convert #1 `dirname #1`/`basename #1 .tif`.png}

%\title{Circle of Life}
%\author{Patrick Misteli, Ruben K{\"a}lin}
%\date{date} 

\begin{document}



\thispagestyle{empty}

\begin{center}
\includegraphics[width=5cm]{ETHlogo.eps}

\bigskip


\bigskip


\bigskip


\LARGE{ 	Lecture with Computer Exercises:\\ }
\LARGE{ Modelling and Simulating Social Systems with MATLAB\\}

\bigskip

\bigskip

\small{Project Report}\\

\bigskip

\bigskip

\bigskip

\bigskip


\begin{tabular}{|c|}
\hline
\\
\textbf{\LARGE{Circle Of Life}}\\
%\textbf{\LARGE{}}\\
\\
\hline
\end{tabular}
\bigskip
\\
\includegraphics[scale=0.69]{images/circleEdited.png}\label{fig:circle}\\
\bigskip
\LARGE{Patrick Misteli \& Ruben K{\"a}lin}

\bigskip

\bigskip

\bigskip

\bigskip

\bigskip

\bigskip

Zurich\\
May 2014\\

\end{center}



\newpage

%%%%%%%%%%%%%%%%%%%%%%%%%%%%%%%%%%%%%%%%%%%%%%%%%

\newpage
\section*{Agreement for free-download}
\bigskip


\bigskip


\large We hereby agree to make our source code for this project freely available for download from the web pages of the SOMS chair. Furthermore, we assure that all source code is written by ourselves and is not violating any copyright restrictions.

\begin{center}

\bigskip


\bigskip


\begin{tabular}{@{}p{3.3cm}@{}p{6cm}@{}@{}p{6cm}@{}}
\begin{minipage}{3cm}

\end{minipage}
&
\begin{minipage}{6cm}
\vspace{2mm} \large Patrick Misteli

 \vspace{\baselineskip}

\end{minipage}
&
\begin{minipage}{6cm}

\large Ruben K{\"a}lin

\end{minipage}
\end{tabular}


\end{center}
\newpage

%%%%%%%%%%%%%%%%%%%%%%%%%%%%%%%%%%%%%%%



% IMPORTANT
% you MUST include the ETH declaration of originality here; it is available for download on the course website or at http://www.ethz.ch/faculty/exams/plagiarism/index_EN; it can be printed as pdf and should be filled out in handwriting


%%%%%%%%%% Table of content %%%%%%%%%%%%%%%%%

\tableofcontents

\newpage

%%%%%%%%%%%%%%%%%%%%%%%%%%%%%%%%%%%%%%%



\section{Abstract}
In the film "Lion King" from Disney the main character's father explains to his son the concept of what he calls "the circle of life" and how all animals in the food chain hierarchy are needed in order for every species to survive. He abstracts the concept by explaining that lions become grass after they die. This grass is eaten by the antelopes, which are again eaten by the lions to complete the circle. We wanted to see if this concept is sound and said three organisms are able to survive within cellular automaton simulation.

\section{Individual contributions}
Patrick was responsible for the model, the parameters and the implementation in matlab thereof. Ruben polished single aspects of the simulation and was responsible for the performance of the algorithm. Furthermore Ruben was responsible for the inclusion of the Lotka Volterra equation and finding papers thereof.

\section{Introduction and Motivations}
Having a clear food chain hierarchy defined by the given "circle of life", we were not only interested in whether it is possible to create a stable system, but also which parameters effect the system in what way. Changing a minor parameter, for example how fast animals are able to reproduce or how hungry they are could have a major effect on the balance. Is it possible for the system to remain stable if we were to remove one of the organisms? To the end of the mentioned film "Lion King" the balance is destroyed when the lions start disrespecting the circle of life by overfeeding. Our model should be able to reproduce this effect when making the simulated lions "very hungry".

\section{Description of the Model}
\subsection{States}
Starting with Conway's Game of Life \cite{gameOfLife} a cell should not only have two states (active or inactive) it should now have the following four states representing all organisms:

\begin{description}
  \item[Nothing or Inactive (Black)] \hfill \\
	This represents a dead land cell where no living organism is located
  \item[Grass (Green)] \hfill \\
	This represents a living grass organism
  \item[Antelope (Brown)]  \hfill \\
	This represents a living antelope organism
  \item[Lion (Red)] 
   \hfill \\This represents a living lion organism
\end{description}

\subsection{Attributes}
Furthermore each of the four cell types must have parametrizable attributes in order for the extended rules to work. 
Table \ref{tab:Properties} lists all attributes and describes their meaning.
\begin{table}[htbp]
\begin{tabular}{l|p{11cm}}\label{tab:Properties}
Name [Value Range]& Description\\
\hline 
\hline 
Type [1,4]& The number between 1 and 4 represents "Nothing", "Grass", "Antelope" and "Lion" respectively  \\ 
\hline 
PreyType [1,4] & Type of the organism which the current organism eats \\ 
\hline 
Becomes [1,4]& Type to become after a natural death (Dying of age or hunger) \\ 
\hline 
FoodDigest [0,inf]& Amount of food that is digested in one timestep\\ 
\hline 
Stomach [0,inf]& Amount of food currently in organism. This decreases every timestep (by the number defined in FoodDigest) and increases when the organism found food\\ 
\hline 
MaxStomach [0,inf]& Maximal amount of food an organism will eat. Organism will only eat if its stomach is below this number. A high number will cause the organism to eat anything it finds causing overfeeding while a low number will cause the organism to live on a diet \\ 
\hline 
DeathProb [0,1]& The probability of dying by age at the end of a the timestep. Other death causes are discussed in section \ref{tab:deathCauses} \\ 
\hline 
Fatness [0,inf]& The number of predators that can be fed eating this organism (Note: as soon as one predator takes a bite the current organism will become Nothing at the end of the timestep)\\ 
\hline 
Alive {0,1}& 1 if alive, 0 if a predator has bitten this organism\\
\hline 
MinStomachRep [0,inf]& Minimal value of Stomach needed to reproduce (Reproduction explained in section \ref{tab:reproduction}) \\
\hline 
RepProb [0,1]& Probability to reproduce when allowed (Reproduction explained in section \ref{tab:reproduction})\\
\hline 
IsOffspring {0,1}& 1 if newborn, 0 if adult\\
\end{tabular}
\end{table}

\subsection{Rules}
\label{tab:reproduction}
A standard Conway's Game of Life only has four simple rules (which can be summarized to even less than four). Since our model has more complexity by having more than 2 states and certain attributes to take in account we adapted the rules to get as close to nature as possible. As in game of life we inspect our neighbourhood and act accordingly. 
\subsubsection{Neighbourhood}
We started off with the standard Moore-neighborhood but extended it by four additional cells attached to the northern, southern, western and eastern neighbour. This extension allows an organism so "see" further and the surrounding map


\subsection{Death Causes}
In our model there are multiple ways a single organism can cease to exist. Depending on the cause the outcome differs too. 
\\
\begin{tabular}{l|p{11cm}}\label{tab:deathCauses}
Death Cause & Description \\ 
\hline 
\hline 
Age & At each timestep a random number between 0 and 1 is generated. If this number is smaller than the specified death probability the organism dies of age\\ 
\hline 
Hunger & An organism with an empty stomach dies of hunger \\ 
\hline 
Eaten & Prey dies at the end of the day if bitten by predator \\  
\end{tabular} 


Known limits
Why antelopes do not become grass?
Die because of location and inability to move
Cannot walk
Neighborhood

\section{Implementation}
We implemented a cell based system that simulates the habitat of three different species.
We compared the population results obtained form the simulation with the Lotka-Volterra population curves for three species, \cite{lotkaVolterraThreeSpecies}. The results of this comparison enabled us to gain better understanding of the equilibria and the oscillating behaviour of the different populations in our simulation.

\subsection{Lotka Volterra}
The population of different organisms has been thoroughly studied. A.J. Lotka and V. Volterra found formulas describing the population number of two species, \cite{lotka}, \cite{volterra}. One species is a predator and the other species is the prey. The formulas nicely model the dynamics of the population. A large number of predators results in a decrease in the number of prey organisms, a small number of predators allows the prey organisms to increase in number. On the other hand, the predators are fed by the prey, so predators only flourish when there are enough organisms to prey upon. When the number of prey organism becomes too little, predators die of hunger. These rules are nicely reflected in the formulas of Lotka and Volterra, \cite{lotkaVolterra}.\\

\begin{equation}
\begin{split}
\frac{dx}{dt} = ax-bxy \\ 
\frac{dy}{dt} = cxy-dy
\end{split}
\end{equation}
The reasoning about the population count of organisms forming a food chain has been extended to three species, \cite{lotkaVolterraThreeSpecies}. 

\begin{equation}
\begin{split}
\frac{dx}{dt} = ax-bxy \\ 
\frac{dy}{dt} = -cy+dxy-eyz \\ 
\frac{dz}{dt} = -fz+gyz
\end{split}
\end{equation}

\section{Simulation Results and Discussion}
Death by Age $\rightarrow$ increase reproduction Rate
Remove one species what happens
Location of animals have impact on survival, not only numbers
Cool pictures

\section{Summary and Outlook}
Answer questions

\appendix

\section{Appendix}
The title image was taken from \cite{titleImage}.

\section{Code}
%the final source code goes here

\bibliographystyle{plain}
\bibliography{bibliography}
\addcontentsline{toc}{chapter}{Bibliography}

\end{document}  